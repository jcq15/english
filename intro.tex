Air pollution has been extensively studied in recent years. Many recent studies have focused on the impact of air pollution. Some researcher clearly shows that air pollution does harm to human health. Hazardous chemicals escape to the environment by a number of natural and/or anthropogenic activities, it has both acute and chronic effects on human health and affect a number of different systems and organs (Kampa \& Castanas, 2007). Air pollution causes many  respiratory diseases. Long-term exposure to combustion-related fine particulate air pollution is an important environmental risk factor for cardiopulmonary and lung cancer mortality (Pope et al., 2002). 

In order to deal with issues of why the air pollution makes so many effects on human health, a lot of researchers work on it and try to come up with the principles of pathogenicity. Many researchers have studied the principle of lung cancer caused by air pollution, and have got some results. Some ultra-fine particles, which size between $10^6nm/mL$, are able to provoke alveolar inflammation, with release of mediators capable, in susceptible individuals, of causing exacerbations of lung disease and of increasing blood coagulability (A. Seaton, D. Godden, W. MacNee, K. Donaldson, 1995). 

The results of these studies have made a great contribution to our understanding of air pollution and human health, they show the impact and the principles of pathogenicity for us. However, little work attempted to the impact of air pollution on people’s daily activities. We need to know how air pollution affects people's daily life. For example, how many people will give up outdoor activity if it is a polluted day? In this paper we give preliminary results for the impact. Specifically, we are going to talk about the difference of college students’ activity in Tsinghua University between clean air and polluted air. 
