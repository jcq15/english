\subsection{Questionnaire and participants}
We used questionnaire to survey the specific impact which air pollution caused to students in Tsinghua. The questionnare was devided into two parts: basic information and impact on daily activity and impact on mood. In the questionnare, question 1~6 were basic information about participants, question 7 were about daily activities. The second part was the most important part, we set six air pollution level by Air Quality Index (AQI). Participants were supposed to answer how high AQI would cause them stop the activity. In this way, we could know whether and how much the air pollution impact our daily activities. It was about these activities:

\begin{enumerate}
\item go to class in classroom
\item study at library
\item go to PE class
\item some relaxing outside sports (walk, ride .etc)
\item some intense sports (football, basketball .etc)
\item meet classmates outside
\item run outside
\item study in dormitory
\end{enumerate}

Although we wanted to work on all people around the world, however, limited by many factors, we could only do the survey in my school. So the participants of the study were \_ undergraduate students enrolled in Tsinghua University (notes: we don't know how many participants now). Participants are evenly distributed across the various faculties.

\subsection{Data processing}
For each activity, we knew the relation between AQI and the rate of student who wanted to do this activity, so we could use the Pearson product-moment correlation coefficient (PPMCC) to measure the relevance of the two variables. Then we did significant test to verification if the results are reasonable.